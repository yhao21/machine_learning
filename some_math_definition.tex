
\documentclass[12pt]{article}
\usepackage{amsmath}
\DeclareMathOperator*{\argmin}{arg\,min} % thin space, limits underneath in displays
\DeclareMathOperator*{\argmax}{arg\,max} % thin space, limits underneath in displays
\newtheorem{thm}{Theorem}
\usepackage{amssymb}
\usepackage{amsfonts}
\usepackage{mathrsfs}
\usepackage{bm}
\usepackage{indentfirst}
\setlength{\parindent}{0em}
\usepackage[margin=1in]{geometry}
\usepackage{graphicx}
\usepackage{setspace}
\doublespacing
\usepackage[flushleft]{threeparttable}
\usepackage{booktabs,caption}
\usepackage{float}
\usepackage{graphicx}

\usepackage{import}
\usepackage{xifthen}
\usepackage{pdfpages}
\usepackage{transparent}

\newcommand{\incfig}[1]{%
\def\svgwidth{\columnwidth}
\import{./figures/}{#1.pdf_tex}
}




\title{Some Basic Math}
\author{Synferlo}
\date{Apr. 7, 2021}


\begin{document}
\maketitle
\newpage



1. $ \argmax $ vs. $ \max_{\substack{\\}}  $

$ \max_{\substack{\\}}f(x)  $ is the maximum value (if it exists) of
$ f(x) $ as $ x $ varies through some domain.

$ \argmax f(x) $ is the value of $ x $ at which this maximum is attained.


In short, $ \argmax $ is the value of $ x $, while $ \max_{\substack{\\}}  $
is the value of the function $ f(x) $.









\end{document}

